%%%
%%% The Mite virtual machine
%%%
%%% (c) Reuben Thomas 2000–2016
%%%

%%% FIXME: Work out what the possible error conditions are in verified code & list them.


\documentclass[english]{scrartcl}
%%% Put usepackages one per line for HEVEA
\usepackage[LY1]{fontenc}
\usepackage{babel}
\usepackage{array}
\usepackage{amstext}
\usepackage{booktabs}
\usepackage{url}
%\usepackage[active]{srcltx}

%%% Settings for this document
\urlstyle{rm}

%%% Macros for this document
\newcommand{\cont}{\noindent}
\newcommand{\synfont}{\sffamily}
\newcommand{\syn}[1]{\ensuremath{\text{\synfont #1}}}
\newcommand{\norm}[1]{\textrm{#1}}

%%% Centred tables
\newenvironment{ctabular}[1]%
{\begin{center}\begin{tabular}{#1}}%
    {\end{tabular}\end{center}}
\newcommand{\ctablecaption}{}
\newenvironment{ctable}[3][tbp]%
{\renewcommand{\ctablecaption}{#2}%
  \begin{table}[#1]\begin{center}\begin{tabular}{#3}}%
      {\end{tabular}\caption{\ctablecaption}\end{center}\end{table}}

%%% Instruction tables
\newenvironment{insttab}[1]
{\begin{ctable}{#1}{>{\synfont}p{13ex}p{44ex}}\toprule}
  {\bottomrule\end{ctable}}


\begin{document}

\title{The Mite virtual machine}
\author{Reuben Thomas\\\url{rrt@sc3d.org}\\\url{http://rrt.sc3d.org/}}
\date{27th July 2016} \maketitle % FIXME: date should be that of spec.lua



\section{Introduction}

Mite is a general-purpose virtual machine. It is designed to be
capable of efficient implementation, by interpretation or compilation,
from a binary-portable object format. It has a flat linear memory, and
single-threaded, non-interruptable execution. It is designed to
interwork with native machine code.



\section{Parameters}

Mite is parametrized on the following quantities:

\begin{description}
\item[$W$]number of bytes in a word

\item[$R$]the number of registers ($R=2^n, 2<n\leq 8$)

\item[$S$]direction of stack growth ($-1$ for an ascending stack, $1$
  for a descending stack)
\end{description}



\section{Registers and memory}

Registers are word-sized. Mite has $R$ registers \syn{1} to $R$, a
program counter \syn{P}, a stack pointer \syn{S}, a frame pointer
\syn{F}, and a temporary register, \syn{T}. The stack is a LIFO
stack; \syn{S} points to the top-most item, and \syn{F} points to the
bottom of the current frame.

The memory $M$ is an array of bytes. The stack resides in memory.
$M_s(a)$ denotes the $s$ bytes in memory starting at byte $a$; $a$
must be a multiple of $W$, and $s$ (the ``size'') must be a power of
$2$ between $1$ and $W$ inclusive, or $w$, which represents $W$.
Within a word, bytes may be stored in big or little-endian order.
Two's complement number representation is used.



\section{Execution}

Mite repeatedly loads the instruction at \syn{P}, makes \syn{P} point
to the next instruction, and executes the loaded instruction. Program
addresses need not be addresses in $M$.



\section{Instructions}

In the syntax, each operand is denoted by a letter subscripted by a
number. The letter indicates the sort of operand required, as shown in
table~\ref{optypetab}; the number shows the number of the operand.

\begin{ctable}{Operand types\label{optypetab}}{>{$}c<{$}l}
  \toprule \input{opTypes} \bottomrule
\end{ctable}

$\{P\}$ has the value $1$ if the predicate $P$ is true, and $0$
otherwise. $\rho$ is a random value.


\subsection{Labels}

Labels are defined by the \syn{lab} instruction, shown in
table~\ref{labtab}.

\begin{insttab}{Label definition\label{labtab}}
  \input{instLab}
\end{insttab}

There types of label are: branch labels, denoted \syn{b}, subroutine
labels, denoted \syn{s}, data labels, denoted \syn{d}, and function
labels, denoted \syn{f}. A label is written as its name. A name is a
string of letters, digits and underscores, and may not start with a
digit; names must be unique within a translation unit.

The value of a branch, subroutine or function label is the address of
the instruction immediately following it. A data label's value is the
address of the first datum stored after it by a data instruction.

Branch labels have no effect when they are executed; subroutine, data
and function labels may not be executed.


\subsection{Computation}

\begin{insttab}{Computational instructions\label{comptab}}
  \input{instComp}
\end{insttab}

The computational instructions are shown in table~\ref{comptab}.

Immediate constants are of the form
[\syn{e}][\syn{s}][\syn{w}]$n$[\syn{>\/>}$r$]. $n$ (and $r$, if
present) is an integer. If \syn{e} (for ``endianness'') is present,
$n$ is subtracted from $W$ on a big-endian Mite, or left unaltered
otherwise; then if \syn{s} is present, it is multiplied by $S$, and if
\syn{w} is present, it is multiplied by $W$. The final value is
truncated to $8W$ bits, then if \syn{>\/>}$r$ is present, it is
rotated by $r$ places, to the right if positive, and to the left if
negative.


\subsection{Data}

The data instructions, shown in table~\ref{datatab}, allow literal
data to be included in object code; they may not be executed. Data
between two labels are stored contiguously in $M$.

\begin{insttab}{Data instructions\label{datatab}}
  \input{instData}
\end{insttab}

\syn{lit} causes its operand to be truncated to a word and stored in
the next word of memory. \syn{litl} causes the values of the given
label to be stored in the next word of memory.

\syn{space} causes the given number of zero words to be stored in
consecutive locations.


% \subsection{C functions}

% The instructions shown in table~\ref{ccalltab} allow C functions to be
% both implemented and called by Mite code.

% \begin{insttab}{C function instructions\label{ccalltab}}
%   \input{instCcall}
% \end{insttab}

% An argument type is \syn{r} for an integer register, \syn{f} for a
% float register or \syn{b}$n$ for a memory block of $n$ words.

% In an \syn{arg} instruction, a memory block is given by a register
% holding the address of the block. Arguments for a function call are
% given in order from right to left.

% There must be no branch labels or branches between a \syn{func}
% instruction and the corresponding \syn{callf}, \syn{callfr} or
% \syn{callfn}. When \syn{getret} is used it must immediately follow the
% corresponding call instruction.



\section{Object format}

In the description below, hexadecimal numbers are indicated by a
leading ``0x''.

Object code consists of a series of instructions.


\begin{ctable}{Instruction opcodes\label{opcodetab}}{*{3}{>{\synfont}cc!{\hspace{5mm}}}}
  \toprule \norm{\bf Instruction} & \bf Opcode & \norm{\bf
    Instruction} & \bf Opcode & \norm{\bf Instruction} & \bf Opcode \\
  \midrule \input{instOpcode} \bottomrule
\end{ctable}

\subsection{Instructions} \label{objinst}

Instructions are encoded as the opcode followed by the operands, in
numerical order. Instructions that do not end on a $4$-byte boundary
are padded to the next such boundary with zero bytes.

Opcodes, registers and label types are encoded as one byte. The
instruction opcodes are shown in table~\ref{opcodetab}. Opcodes
0x25--0x7f are reserved for future expansion. Registers are encoded by
their number. Label types are encoded as $1$ for \syn{b}, $2$ for
\syn{s}, $3$ for \syn{d} and $4$ for \syn{f}. Argument types are
encoded as one byte for the argument type ($1$ for \syn{r}, $2$ for
\syn{f} or $3$ for \syn{b}$n$) followed, for a \syn{b} type, by a long
number giving the size.

Labels are encoded as a long number (see section~\ref{longnums}).
Labels of each type are numbered consecutively from $1$ from the start
of the translation unit, according to the order in which they are
declared. For a \syn{lab} instruction, only the label type is encoded
(the number is redundant).

Immediate constants are encoded as a byte encoding the modifiers
followed by a byte encoding the rotation (if any), followed by the
basic value encoded as a long number. The modifiers are encoded as
binary flags, as shown in table~\ref{modenctab}; the unused bits are
zeroed.

\begin{ctable}{Constant modifiers encoding\label{modenctab}}{>{\synfont}cc}
  \toprule \bf Modifier & \bf Bit position \\ \midrule
  \textrm{rotation} & 0 \\
  w & 1 \\
  s & 2 \\
  e & 3 \\
  \bottomrule
\end{ctable}


\subsection{Long numbers} \label{longnums}

Long numbers are encoded as follows:

\begin{enumerate}
\item Left-truncate the number to the minimum number of bits in which
  it can be represented.

\item Remove an $8n-1$-bit word from the most significant end of the
  number, where $n$ is the number of bytes left in the current
  instruction word. If there are fewer than $8n-1$ bits in the number
  then sign-extend it to that length first.

\item Add a bit to the most significant end of the word, which should
  be zero if more bits remain in the number, and one otherwise.

\item Store the word, with the bytes in big-endian order.

\item While there are still bits in the number, repeat the following
  steps:

  \begin{enumerate}
  \item Remove a $31$-bit word from the most significant end of the
    number, padding with zeroes if there are fewer than $31$ bits left.

  \item Add a bit to the most significant end of the word, which
    should be zero if more bits remain in the number, and one
    otherwise.

  \item Store the word, with the bytes in big-endian order.
  \end{enumerate}

\end{enumerate}



\section{Acknowledgements}

Martin Richards introduced me to Cintcode~\cite{cintweb}, which
kindled my interest in virtual machines, and led to
Beetle~\cite{beetledis} and an earlier version of Mite~\cite{mite0},
of which the current Mite is a sort of synthesis.
GNU~\emph{lightning}~\cite{lightning} helped inspire me to greater
simplicity, while still aiming for good performance. Alistair Turnbull
has been a fount of criticism for all my work on virtual machines.


\bibliographystyle{plain}
\bibliography{mite}


\end{document}
