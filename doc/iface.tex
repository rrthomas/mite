%
% The Mite translator interface
%
% Reuben Thomas   14/7-23/10/00
%

\documentclass[english]{scrartcl}
\usepackage[LY1]{fontenc}
\usepackage{babel}
\usepackage{array}
\usepackage{booktabs}
\usepackage{url}


% Macros for this document

\newcommand{\absfont}{\sffamily}
\newcommand{\abs}[1]{{\absfont #1}}


\begin{document}

\title{The Mite translator interface}
\author{Reuben Thomas\\\url{rrt@sc3d.org}\\\url{http://sc3d.org/rrt/}}
\date{23rd October 2000}
\maketitle



\section{Introduction}

The Mite translator interface provides a standard way of interacting
with Mite translators, whether implemented as interpreters,
just-in-time compilers, or by some other method. The interface is
described in generic terms; a ISO~C implementation is also given.



\section{C interface}

The C interface is contained in \verb|mite.h|, which imports a
per-implementation header, \verb|mite_i.h|, to define the parameters and
implementation-dependent types, which are suffixed \verb|_i|. To avoid name
clashes, \verb|mite_| is prefixed to all external identifiers. The
interface is actually ANSI~C89 except for the use of \verb|stdint.h| to
obtain \verb|uint32_t|. \verb|CHAR_BIT| must be $8$.



\section{Parameters}

The translator must provide the values it assigns to Mite's
parameters: \abs{g} represents $g$, \abs{s} represents $s$, and
\abs{w} represents $w$. \verb|mite_Word|, \verb|mite_UWord|, \verb|mite_s| and
\verb|mite_g| are defined in \verb|mite_i.h|.



\section{Types}

A Mite translator should provide the following types:

\begin{description}
\item[\abs{UInt32}]an unsigned $32$-bit integer
\item[\abs{Word}]a signed $w$-bit word
\item[\abs{UWord}]an unsigned $w$-bit word
\item[\abs{State}]the state of a computation
\item[\abs{Object}]an object file
\item[\abs{Program}]a translated program
\end{description}


\subsection{State} \label{state}

A \abs{State} has the following members:

\begin{description}
\item[\abs{R}]the register array
\item[\abs{M}]the memory array
\end{description}

\noindent The \abs{S} and \abs{P} registers are therefore accessible via
\abs{R}. There may also be implementation dependent information stored
in the \verb|mite_State_i| element, \verb|_i|.

\noindent In the C implementation, if \verb|M| is \verb|NULL| then $M$ is the
address space of Mite's client.

The following methods may be used to manipulate a \abs{State}:

\begin{description}
\item[\abs{push} (\abs{Word} \abs{w}) : ()]Push \abs{w} on to the stack.
\item[\abs{pop} () : (\abs{Word})]Pop a word from the stack.
\end{description}

\noindent The stack can be statically allocated, or dynamically extended.
\abs{push} can raise \abs{badS} (if the stack overflowed) or
\abs{memoryFull} (if there was no memory to extend it); \abs{pop} can
raise \abs{badS} (to signal stack underflow).


\subsection{Object}

\abs{Object} is an array of $32$-bit words.


\subsection{Program}

\abs{Program} is an abstract type; \verb|mite_Program| is defined by
\verb|mite_i.h|.



\section{Methods} \label{methods}

A translator should provide the following methods:

\begin{description}
\item[\abs{translate (Object o) : Program}]\ \\\raggedright\abs{raises
\{internalError, memoryFull, badInstruction (Word),
badHeader\}}\\translates \abs{o} into a runnable program
\item[\abs{run (Program p, State s) : Word}]\
\\\raggedright\abs{raises \{internalError, memoryFull, badP, badS,
badAddress, divisionByZero, badShift, badInstruction (Ref
UInt32)\}}\\runs program \abs{p} starting in state \abs{s}, and
returns the value passed to the \abs{halt} instruction
\end{description}

In the C interface, \verb|translate| uses \verb|errp| to return the number of a
\abs{badInstruction} error; in \verb|run| the address is in \abs{P}
(\verb|s.R[0xff]|). The \verb|jmp_buf| supplied to each function is used as the
destination of \verb|longjmp()| in the event of an exception being raised.



\section{Errors}

The following errors can be raised by the translator, as detailed above:

\begin{description}
\item[\abs{internalError}]internal error in the translator
\item[\abs{memoryFull}]the translator has run out of memory
\item[\abs{badP}]\abs{P} out of range
\item[\abs{badS}]\abs{S} out of range
\item[\abs{badAddress}]address operand out of range or unaligned
\item[\abs{divisionByZero}]division by zero
\item[\abs{badShift}]shift amount greater than $8w$
\item[\abs{badInstruction (UInt 32)}]there is an invalid instruction
in the object module; its address is returned
\item[\abs{badHeader}]the object module's header is invalid
\end{description}


\end{document}
