%
% mit, the Mite translator
%
% Reuben Thomas   20/5/00-3/4/01
%

% the line below makes lmk -pvc work
%%
% mit, the Mite translator
%
% Reuben Thomas   20/5/00-3/4/01
%

% the line below makes lmk -pvc work
%%
% mit, the Mite translator
%
% Reuben Thomas   20/5/00-3/4/01
%

% the line below makes lmk -pvc work
%%
% mit, the Mite translator
%
% Reuben Thomas   20/5/00-3/4/01
%

% the line below makes lmk -pvc work
%\input{mit.w}

% Do a special h3sm GNU C version (inspired then by Hohensee building on
% Ertl (et al?))

\documentclass[english]{scrartcl}
\usepackage[LY1]{fontenc}
\usepackage{babel}
\usepackage{lypslatex}
\usepackage{array}
\usepackage{booktabs}
\usepackage{url}


% Macros for this document

\newcommand{\absfont}{\sffamily}
\newcommand{\abs}[1]{{\absfont #1}}



\begin{document}

\title{mit, the Mite translator}
\author{Reuben Thomas\\\url{rrt@sc3d.org}\\\url{http://sc3d.org/rrt/}}
\date{3rd April 2001}
\maketitle



\section{Introduction}

mit, the \textbf{Mi}te \textbf{t}ranslator, is written in ISO~C, and can be
used stand-alone or embedded in other programs.


\subsection{License}

mit is supplied as is, under BSD license:

\begin{quote}
Copyright \copyright 2001 Reuben Thomas

Permission is hereby granted, free of charge, to any person obtaining
a copy of this software and associated documentation files (the
``Software''), to deal in the Software without restriction, including
without limitation the rights to use, copy, modify, merge, publish,
distribute, sublicense, and/or sell copies of the Software, and to
permit persons to whom the Software is furnished to do so, subject to
the following conditions:

The above copyright notice and this permission notice shall be
included in all copies or substantial portions of the Software.

THE SOFTWARE IS PROVIDED ``AS IS'', WITHOUT WARRANTY OF ANY KIND,
EXPRESS OR IMPLIED, INCLUDING BUT NOT LIMITED TO THE WARRANTIES OF
MERCHANTABILITY, FITNESS FOR A PARTICULAR PURPOSE AND NONINFRINGEMENT.
IN NO EVENT SHALL THE AUTHORS OR COPYRIGHT HOLDERS BE LIABLE FOR ANY
CLAIM, DAMAGES OR OTHER LIABILITY, WHETHER IN AN ACTION OF CONTRACT,
TORT OR OTHERWISE, ARISING FROM, OUT OF OR IN CONNECTION WITH THE
SOFTWARE OR THE USE OR OTHER DEALINGS IN THE SOFTWARE.
\end{quote}


\subsection{System requirements}

mit should work on any big or little-endian system with an ISO~C
compiler and the standard libraries, provided:

\begin{itemize}
\item two's complement number representation is used
\item the compiler performs arithmetic right shifts on signed types
\item all pointers are the same size
\end{itemize}

\noindent I'd really like to know of any problems, whether because
your system doesn't meet these criteria, or because of a bug in mit.



\section{Interpreter}

The interpreter is contained in \verb|interp.c|. Some functions are
compiled twice, once with \verb|MITE_VERIFY| defined and once without.
The verified versions have some extra checks for validity of the
object code added, which are not needed during interpretation, once
the code has been checked; they have the prefix \verb|ver_| added to
their names by the \verb|MITE_PREFIX| macro.

The implementation-specific header is \verb|mite_i.h|. It contains the
implementation-dependent parameters and types.



\section{Utilities}

Mite uses a few utility macros. \verb|min()| and \verb|max()| are the
usual minimum and maximum operators; \verb|raise()| is a shorthand for
raising an exception. \verb|seta()| sets array element \verb|a[i]| to
\verb|e|$+$\verb|o|, incrementing \verb|i| and checking that it does
not exceed the size \verb|n| of the array. \verb|alloc()| is a wrapper
for \verb|malloc()| that raises an exception if it runs out of memory.



\section{Parameters}

\verb|mite_i.h| is included by \verb|mite.h| to define the
implementation-specific parameters and types. mit represents the Mite
word by \verb|intptr_t| and the unsigned word by \verb|uintptr_t|;
there are $8$ general registers by default (though this can be changed
by altering \verb|mite_g|'s definition), and the stack is descending.

The endianness of the current machine is held in \verb|endianness|,
which is simply a \verb|uint32_t| set to $1$. It is read via the macro
\verb|LITTLE_ENDIAN|, which looks at the first byte, which will be $1$
on a little-endian machine, and $0$ on a big-endian machine.



\section{Registers and stack}

Some macros are used to access registers and the stack, which depend
on \verb|R| being a register array. \verb|P_REG| is the register
number of the \abs{P} register; \verb|P| gives the value of \abs{P} as
a pointer to an instruction; \verb|Pp| gives the address of \verb|P|
as a pointer to \verb|miteObject|, which can be passed to
\verb|intg()| and \verb|num()| (section~\ref{numbers}) while reading
virtual code. \verb|setP()| is used to assign to \verb|P| (because
ANSI~C forbids casting l-values, and \verb|P| is actually a member of
an array of \verb|miteWord|s, not a \verb|uint32_t *|).

Similarly, \verb|S_REG| is the register number of \abs{S}, \verb|S|
gives \abs{S}'s value as a \verb|miteWord *|, and \verb|setS()| is
used to assign to \verb|S|. \verb|checkS()| is called by \verb|setS()|
and elsewhere to check that \abs{S}'s value is valid every time it is
updated; if not, \verb|mite_badS| is raised.

The stack is stored in a \verb|malloc|ed block, which is auto-extended
using the \verb|stack_extend()| macro, which is passed the
\verb|mite_State| whose stack it should extend (including a trailing
\verb|.| or \verb|->| to indicate the method of access to the
members). This macro is also used to initialise the stack, by setting
the state's \verb|S_base| to \verb|NULL|, which causes the call to
\verb|realloc()| to allocate a new block of memory. \verb|extendS| is
used during interpretation before each operation that pushes to the
stack, to extend it when full.



\section{Numbers} \label{numbers}

These two functions read numbers from a \verb|miteObject *|, updating
the pointer once the number has been read.

\verb|intg()| reads an encoded integer, gathering words until the top
bit is zero, and taking the sign from the first word. When verifying,
it takes as its second parameter a pointer to the end of the object
code and checks that it does not exceed this limit.

\verb|num()| reads a number with possible use of $e$, $s$, $w$ and rotation.
Because it calls \verb|intg()|, it too must have a verified version.



\section{Types}

\verb|miteProgram| holds all the data needed to interpret a program.
\verb|code| points to the start of the object code in the original
module, and \verb|cend| points to the end of the object code;
\verb|data| points to the copied out data, and \verb|dlen| is its
length; \verb|bl|, \verb|sl| and \verb|dl| point to arrays containing
the addresses of the labels, and \verb|b|, \verb|s| and \verb|d| give
the number of labels of each types.

\verb|mite_State_i| holds the base of the stack (\verb|S_base|), and
the amount of memory currently allocated for it (\verb|S_size|).



\section{Translation}

Before a program is interpreted, it is scanned to find all the labels,
which are stored in arrays for faster access during interpretation,
and the literal data is written into a separate memory block so that
it can easily be read and written. At the same time, the program is
verified to check that all the instructions and their operands are
valid.

\verb|mite_translate()| translates a program for interpretation,
verifying that it is valid in the process.

The number of labels of each type is now checked against the numbers
given in the header; the checks made while setting the values of
labels ensured that there can't be more labels than required, but
there might be fewer.

The current value of \verb|p| is stored as the value of the new label,
according to its type, then the number of labels of the current type
is checked to make sure it doesn't overflow.


\subsection{Verifying and copying data}

Next, the object code is scanned again. The data are copied out, and
the types of label operands are checked.

Finally, the program is returned.


\subsection{Reading data}

For a \abs{lit}, the number of operands, \verb|ops|, is read, and that
number of literal numbers is copied into the data block.

\abs{litl} can refer to the three types of label, whose addresses were
calculated in the previous pass over the object code.

\abs{space} requires the given number of words to be zero-filled.



\section{Execution}

\subsection{Operands}

Macros are provided to read operands of each type. The special type
\verb|_| is for unused operand places.

Given an instruction's opcode and operand types, \verb|OP()| declares
the case label and reads the instruction's operands. The \verb|break|
is placed before to terminate the previous case, as the code for the
instruction immediately follows. So that the first use of \verb|OP()|
in the switch does not cause a compiler warning, the \verb|default|
case is placed just before the first call of \verb|OP()|.


\subsection{Interpretation loop}

\verb|mite_run()| repeatedly loads, decodes and executes instructions
in an endless loop. The function is exited by a \abs{ret} instruction
when the stack is empty, or via \verb|raise()| if an error occurs.

Instructions are dispatched by a switch on the opcode.


\section{Instruction actions} \label{actions}

The default instruction actions are below. The \verb|OP| macro need
not do any checking of the arguments, as the code being interpreted is
known to be correct. Only run-time errors need be checked. Note that
there is no case for \abs{lab}, which therefore raises
\abs{badInstruction} if it is ever executed. The preprocessing step
sets the address of branch and subroutine labels to the instruction
after the \abs{lab} instruction, so no \abs{lab} instruction should
ever actually be executed.



\section{Standalone interpreter}

The \verb|main()| function allocates the stack, prepares the program,
and then starts it by calling \verb|mit_run()|, whose return value is
used as mit's exit code.



\section{Embedding mit}

mit can be embedded in a C program by compiling \verb|interp_main.c|
into the program. Since the interpreter has no global state, it can be
compiled as a shared library, and is thread-safe.



\section{Acknowledgements}

mit draws on implementation techniques developed for my earlier Beetle
virtual processor~\cite{beetledis}. Martin Richards introduced me to
virtual machine interpreter implementation techniques via his Cintcode
system~\cite{cintweb}.



\bibliographystyle{plain}
\bibliography{rrt,vm}

\end{document}


% Do a special h3sm GNU C version (inspired then by Hohensee building on
% Ertl (et al?))

\documentclass[english]{scrartcl}
\usepackage[LY1]{fontenc}
\usepackage{babel}
\usepackage{lypslatex}
\usepackage{array}
\usepackage{booktabs}
\usepackage{url}


% Macros for this document

\newcommand{\absfont}{\sffamily}
\newcommand{\abs}[1]{{\absfont #1}}



\begin{document}

\title{mit, the Mite translator}
\author{Reuben Thomas\\\url{rrt@sc3d.org}\\\url{http://sc3d.org/rrt/}}
\date{3rd April 2001}
\maketitle



\section{Introduction}

mit, the \textbf{Mi}te \textbf{t}ranslator, is written in ISO~C, and can be
used stand-alone or embedded in other programs.


\subsection{License}

mit is supplied as is, under BSD license:

\begin{quote}
Copyright \copyright 2001 Reuben Thomas

Permission is hereby granted, free of charge, to any person obtaining
a copy of this software and associated documentation files (the
``Software''), to deal in the Software without restriction, including
without limitation the rights to use, copy, modify, merge, publish,
distribute, sublicense, and/or sell copies of the Software, and to
permit persons to whom the Software is furnished to do so, subject to
the following conditions:

The above copyright notice and this permission notice shall be
included in all copies or substantial portions of the Software.

THE SOFTWARE IS PROVIDED ``AS IS'', WITHOUT WARRANTY OF ANY KIND,
EXPRESS OR IMPLIED, INCLUDING BUT NOT LIMITED TO THE WARRANTIES OF
MERCHANTABILITY, FITNESS FOR A PARTICULAR PURPOSE AND NONINFRINGEMENT.
IN NO EVENT SHALL THE AUTHORS OR COPYRIGHT HOLDERS BE LIABLE FOR ANY
CLAIM, DAMAGES OR OTHER LIABILITY, WHETHER IN AN ACTION OF CONTRACT,
TORT OR OTHERWISE, ARISING FROM, OUT OF OR IN CONNECTION WITH THE
SOFTWARE OR THE USE OR OTHER DEALINGS IN THE SOFTWARE.
\end{quote}


\subsection{System requirements}

mit should work on any big or little-endian system with an ISO~C
compiler and the standard libraries, provided:

\begin{itemize}
\item two's complement number representation is used
\item the compiler performs arithmetic right shifts on signed types
\item all pointers are the same size
\end{itemize}

\noindent I'd really like to know of any problems, whether because
your system doesn't meet these criteria, or because of a bug in mit.



\section{Interpreter}

The interpreter is contained in \verb|interp.c|. Some functions are
compiled twice, once with \verb|MITE_VERIFY| defined and once without.
The verified versions have some extra checks for validity of the
object code added, which are not needed during interpretation, once
the code has been checked; they have the prefix \verb|ver_| added to
their names by the \verb|MITE_PREFIX| macro.

The implementation-specific header is \verb|mite_i.h|. It contains the
implementation-dependent parameters and types.



\section{Utilities}

Mite uses a few utility macros. \verb|min()| and \verb|max()| are the
usual minimum and maximum operators; \verb|raise()| is a shorthand for
raising an exception. \verb|seta()| sets array element \verb|a[i]| to
\verb|e|$+$\verb|o|, incrementing \verb|i| and checking that it does
not exceed the size \verb|n| of the array. \verb|alloc()| is a wrapper
for \verb|malloc()| that raises an exception if it runs out of memory.



\section{Parameters}

\verb|mite_i.h| is included by \verb|mite.h| to define the
implementation-specific parameters and types. mit represents the Mite
word by \verb|intptr_t| and the unsigned word by \verb|uintptr_t|;
there are $8$ general registers by default (though this can be changed
by altering \verb|mite_g|'s definition), and the stack is descending.

The endianness of the current machine is held in \verb|endianness|,
which is simply a \verb|uint32_t| set to $1$. It is read via the macro
\verb|LITTLE_ENDIAN|, which looks at the first byte, which will be $1$
on a little-endian machine, and $0$ on a big-endian machine.



\section{Registers and stack}

Some macros are used to access registers and the stack, which depend
on \verb|R| being a register array. \verb|P_REG| is the register
number of the \abs{P} register; \verb|P| gives the value of \abs{P} as
a pointer to an instruction; \verb|Pp| gives the address of \verb|P|
as a pointer to \verb|miteObject|, which can be passed to
\verb|intg()| and \verb|num()| (section~\ref{numbers}) while reading
virtual code. \verb|setP()| is used to assign to \verb|P| (because
ANSI~C forbids casting l-values, and \verb|P| is actually a member of
an array of \verb|miteWord|s, not a \verb|uint32_t *|).

Similarly, \verb|S_REG| is the register number of \abs{S}, \verb|S|
gives \abs{S}'s value as a \verb|miteWord *|, and \verb|setS()| is
used to assign to \verb|S|. \verb|checkS()| is called by \verb|setS()|
and elsewhere to check that \abs{S}'s value is valid every time it is
updated; if not, \verb|mite_badS| is raised.

The stack is stored in a \verb|malloc|ed block, which is auto-extended
using the \verb|stack_extend()| macro, which is passed the
\verb|mite_State| whose stack it should extend (including a trailing
\verb|.| or \verb|->| to indicate the method of access to the
members). This macro is also used to initialise the stack, by setting
the state's \verb|S_base| to \verb|NULL|, which causes the call to
\verb|realloc()| to allocate a new block of memory. \verb|extendS| is
used during interpretation before each operation that pushes to the
stack, to extend it when full.



\section{Numbers} \label{numbers}

These two functions read numbers from a \verb|miteObject *|, updating
the pointer once the number has been read.

\verb|intg()| reads an encoded integer, gathering words until the top
bit is zero, and taking the sign from the first word. When verifying,
it takes as its second parameter a pointer to the end of the object
code and checks that it does not exceed this limit.

\verb|num()| reads a number with possible use of $e$, $s$, $w$ and rotation.
Because it calls \verb|intg()|, it too must have a verified version.



\section{Types}

\verb|miteProgram| holds all the data needed to interpret a program.
\verb|code| points to the start of the object code in the original
module, and \verb|cend| points to the end of the object code;
\verb|data| points to the copied out data, and \verb|dlen| is its
length; \verb|bl|, \verb|sl| and \verb|dl| point to arrays containing
the addresses of the labels, and \verb|b|, \verb|s| and \verb|d| give
the number of labels of each types.

\verb|mite_State_i| holds the base of the stack (\verb|S_base|), and
the amount of memory currently allocated for it (\verb|S_size|).



\section{Translation}

Before a program is interpreted, it is scanned to find all the labels,
which are stored in arrays for faster access during interpretation,
and the literal data is written into a separate memory block so that
it can easily be read and written. At the same time, the program is
verified to check that all the instructions and their operands are
valid.

\verb|mite_translate()| translates a program for interpretation,
verifying that it is valid in the process.

The number of labels of each type is now checked against the numbers
given in the header; the checks made while setting the values of
labels ensured that there can't be more labels than required, but
there might be fewer.

The current value of \verb|p| is stored as the value of the new label,
according to its type, then the number of labels of the current type
is checked to make sure it doesn't overflow.


\subsection{Verifying and copying data}

Next, the object code is scanned again. The data are copied out, and
the types of label operands are checked.

Finally, the program is returned.


\subsection{Reading data}

For a \abs{lit}, the number of operands, \verb|ops|, is read, and that
number of literal numbers is copied into the data block.

\abs{litl} can refer to the three types of label, whose addresses were
calculated in the previous pass over the object code.

\abs{space} requires the given number of words to be zero-filled.



\section{Execution}

\subsection{Operands}

Macros are provided to read operands of each type. The special type
\verb|_| is for unused operand places.

Given an instruction's opcode and operand types, \verb|OP()| declares
the case label and reads the instruction's operands. The \verb|break|
is placed before to terminate the previous case, as the code for the
instruction immediately follows. So that the first use of \verb|OP()|
in the switch does not cause a compiler warning, the \verb|default|
case is placed just before the first call of \verb|OP()|.


\subsection{Interpretation loop}

\verb|mite_run()| repeatedly loads, decodes and executes instructions
in an endless loop. The function is exited by a \abs{ret} instruction
when the stack is empty, or via \verb|raise()| if an error occurs.

Instructions are dispatched by a switch on the opcode.


\section{Instruction actions} \label{actions}

The default instruction actions are below. The \verb|OP| macro need
not do any checking of the arguments, as the code being interpreted is
known to be correct. Only run-time errors need be checked. Note that
there is no case for \abs{lab}, which therefore raises
\abs{badInstruction} if it is ever executed. The preprocessing step
sets the address of branch and subroutine labels to the instruction
after the \abs{lab} instruction, so no \abs{lab} instruction should
ever actually be executed.



\section{Standalone interpreter}

The \verb|main()| function allocates the stack, prepares the program,
and then starts it by calling \verb|mit_run()|, whose return value is
used as mit's exit code.



\section{Embedding mit}

mit can be embedded in a C program by compiling \verb|interp_main.c|
into the program. Since the interpreter has no global state, it can be
compiled as a shared library, and is thread-safe.



\section{Acknowledgements}

mit draws on implementation techniques developed for my earlier Beetle
virtual processor~\cite{beetledis}. Martin Richards introduced me to
virtual machine interpreter implementation techniques via his Cintcode
system~\cite{cintweb}.



\bibliographystyle{plain}
\bibliography{rrt,vm}

\end{document}


% Do a special h3sm GNU C version (inspired then by Hohensee building on
% Ertl (et al?))

\documentclass[english]{scrartcl}
\usepackage[LY1]{fontenc}
\usepackage{babel}
\usepackage{lypslatex}
\usepackage{array}
\usepackage{booktabs}
\usepackage{url}


% Macros for this document

\newcommand{\absfont}{\sffamily}
\newcommand{\abs}[1]{{\absfont #1}}



\begin{document}

\title{mit, the Mite translator}
\author{Reuben Thomas\\\url{rrt@sc3d.org}\\\url{http://sc3d.org/rrt/}}
\date{3rd April 2001}
\maketitle



\section{Introduction}

mit, the \textbf{Mi}te \textbf{t}ranslator, is written in ISO~C, and can be
used stand-alone or embedded in other programs.


\subsection{License}

mit is supplied as is, under BSD license:

\begin{quote}
Copyright \copyright 2001 Reuben Thomas

Permission is hereby granted, free of charge, to any person obtaining
a copy of this software and associated documentation files (the
``Software''), to deal in the Software without restriction, including
without limitation the rights to use, copy, modify, merge, publish,
distribute, sublicense, and/or sell copies of the Software, and to
permit persons to whom the Software is furnished to do so, subject to
the following conditions:

The above copyright notice and this permission notice shall be
included in all copies or substantial portions of the Software.

THE SOFTWARE IS PROVIDED ``AS IS'', WITHOUT WARRANTY OF ANY KIND,
EXPRESS OR IMPLIED, INCLUDING BUT NOT LIMITED TO THE WARRANTIES OF
MERCHANTABILITY, FITNESS FOR A PARTICULAR PURPOSE AND NONINFRINGEMENT.
IN NO EVENT SHALL THE AUTHORS OR COPYRIGHT HOLDERS BE LIABLE FOR ANY
CLAIM, DAMAGES OR OTHER LIABILITY, WHETHER IN AN ACTION OF CONTRACT,
TORT OR OTHERWISE, ARISING FROM, OUT OF OR IN CONNECTION WITH THE
SOFTWARE OR THE USE OR OTHER DEALINGS IN THE SOFTWARE.
\end{quote}


\subsection{System requirements}

mit should work on any big or little-endian system with an ISO~C
compiler and the standard libraries, provided:

\begin{itemize}
\item two's complement number representation is used
\item the compiler performs arithmetic right shifts on signed types
\item all pointers are the same size
\end{itemize}

\noindent I'd really like to know of any problems, whether because
your system doesn't meet these criteria, or because of a bug in mit.



\section{Interpreter}

The interpreter is contained in \verb|interp.c|. Some functions are
compiled twice, once with \verb|MITE_VERIFY| defined and once without.
The verified versions have some extra checks for validity of the
object code added, which are not needed during interpretation, once
the code has been checked; they have the prefix \verb|ver_| added to
their names by the \verb|MITE_PREFIX| macro.

The implementation-specific header is \verb|mite_i.h|. It contains the
implementation-dependent parameters and types.



\section{Utilities}

Mite uses a few utility macros. \verb|min()| and \verb|max()| are the
usual minimum and maximum operators; \verb|raise()| is a shorthand for
raising an exception. \verb|seta()| sets array element \verb|a[i]| to
\verb|e|$+$\verb|o|, incrementing \verb|i| and checking that it does
not exceed the size \verb|n| of the array. \verb|alloc()| is a wrapper
for \verb|malloc()| that raises an exception if it runs out of memory.



\section{Parameters}

\verb|mite_i.h| is included by \verb|mite.h| to define the
implementation-specific parameters and types. mit represents the Mite
word by \verb|intptr_t| and the unsigned word by \verb|uintptr_t|;
there are $8$ general registers by default (though this can be changed
by altering \verb|mite_g|'s definition), and the stack is descending.

The endianness of the current machine is held in \verb|endianness|,
which is simply a \verb|uint32_t| set to $1$. It is read via the macro
\verb|LITTLE_ENDIAN|, which looks at the first byte, which will be $1$
on a little-endian machine, and $0$ on a big-endian machine.



\section{Registers and stack}

Some macros are used to access registers and the stack, which depend
on \verb|R| being a register array. \verb|P_REG| is the register
number of the \abs{P} register; \verb|P| gives the value of \abs{P} as
a pointer to an instruction; \verb|Pp| gives the address of \verb|P|
as a pointer to \verb|miteObject|, which can be passed to
\verb|intg()| and \verb|num()| (section~\ref{numbers}) while reading
virtual code. \verb|setP()| is used to assign to \verb|P| (because
ANSI~C forbids casting l-values, and \verb|P| is actually a member of
an array of \verb|miteWord|s, not a \verb|uint32_t *|).

Similarly, \verb|S_REG| is the register number of \abs{S}, \verb|S|
gives \abs{S}'s value as a \verb|miteWord *|, and \verb|setS()| is
used to assign to \verb|S|. \verb|checkS()| is called by \verb|setS()|
and elsewhere to check that \abs{S}'s value is valid every time it is
updated; if not, \verb|mite_badS| is raised.

The stack is stored in a \verb|malloc|ed block, which is auto-extended
using the \verb|stack_extend()| macro, which is passed the
\verb|mite_State| whose stack it should extend (including a trailing
\verb|.| or \verb|->| to indicate the method of access to the
members). This macro is also used to initialise the stack, by setting
the state's \verb|S_base| to \verb|NULL|, which causes the call to
\verb|realloc()| to allocate a new block of memory. \verb|extendS| is
used during interpretation before each operation that pushes to the
stack, to extend it when full.



\section{Numbers} \label{numbers}

These two functions read numbers from a \verb|miteObject *|, updating
the pointer once the number has been read.

\verb|intg()| reads an encoded integer, gathering words until the top
bit is zero, and taking the sign from the first word. When verifying,
it takes as its second parameter a pointer to the end of the object
code and checks that it does not exceed this limit.

\verb|num()| reads a number with possible use of $e$, $s$, $w$ and rotation.
Because it calls \verb|intg()|, it too must have a verified version.



\section{Types}

\verb|miteProgram| holds all the data needed to interpret a program.
\verb|code| points to the start of the object code in the original
module, and \verb|cend| points to the end of the object code;
\verb|data| points to the copied out data, and \verb|dlen| is its
length; \verb|bl|, \verb|sl| and \verb|dl| point to arrays containing
the addresses of the labels, and \verb|b|, \verb|s| and \verb|d| give
the number of labels of each types.

\verb|mite_State_i| holds the base of the stack (\verb|S_base|), and
the amount of memory currently allocated for it (\verb|S_size|).



\section{Translation}

Before a program is interpreted, it is scanned to find all the labels,
which are stored in arrays for faster access during interpretation,
and the literal data is written into a separate memory block so that
it can easily be read and written. At the same time, the program is
verified to check that all the instructions and their operands are
valid.

\verb|mite_translate()| translates a program for interpretation,
verifying that it is valid in the process.

The number of labels of each type is now checked against the numbers
given in the header; the checks made while setting the values of
labels ensured that there can't be more labels than required, but
there might be fewer.

The current value of \verb|p| is stored as the value of the new label,
according to its type, then the number of labels of the current type
is checked to make sure it doesn't overflow.


\subsection{Verifying and copying data}

Next, the object code is scanned again. The data are copied out, and
the types of label operands are checked.

Finally, the program is returned.


\subsection{Reading data}

For a \abs{lit}, the number of operands, \verb|ops|, is read, and that
number of literal numbers is copied into the data block.

\abs{litl} can refer to the three types of label, whose addresses were
calculated in the previous pass over the object code.

\abs{space} requires the given number of words to be zero-filled.



\section{Execution}

\subsection{Operands}

Macros are provided to read operands of each type. The special type
\verb|_| is for unused operand places.

Given an instruction's opcode and operand types, \verb|OP()| declares
the case label and reads the instruction's operands. The \verb|break|
is placed before to terminate the previous case, as the code for the
instruction immediately follows. So that the first use of \verb|OP()|
in the switch does not cause a compiler warning, the \verb|default|
case is placed just before the first call of \verb|OP()|.


\subsection{Interpretation loop}

\verb|mite_run()| repeatedly loads, decodes and executes instructions
in an endless loop. The function is exited by a \abs{ret} instruction
when the stack is empty, or via \verb|raise()| if an error occurs.

Instructions are dispatched by a switch on the opcode.


\section{Instruction actions} \label{actions}

The default instruction actions are below. The \verb|OP| macro need
not do any checking of the arguments, as the code being interpreted is
known to be correct. Only run-time errors need be checked. Note that
there is no case for \abs{lab}, which therefore raises
\abs{badInstruction} if it is ever executed. The preprocessing step
sets the address of branch and subroutine labels to the instruction
after the \abs{lab} instruction, so no \abs{lab} instruction should
ever actually be executed.



\section{Standalone interpreter}

The \verb|main()| function allocates the stack, prepares the program,
and then starts it by calling \verb|mit_run()|, whose return value is
used as mit's exit code.



\section{Embedding mit}

mit can be embedded in a C program by compiling \verb|interp_main.c|
into the program. Since the interpreter has no global state, it can be
compiled as a shared library, and is thread-safe.



\section{Acknowledgements}

mit draws on implementation techniques developed for my earlier Beetle
virtual processor~\cite{beetledis}. Martin Richards introduced me to
virtual machine interpreter implementation techniques via his Cintcode
system~\cite{cintweb}.



\bibliographystyle{plain}
\bibliography{rrt,vm}

\end{document}


% Do a special h3sm GNU C version (inspired then by Hohensee building on
% Ertl (et al?))

\documentclass[english]{scrartcl}
\usepackage{pxfonts}
\usepackage{babel}
\usepackage{array}
\usepackage{booktabs}
\usepackage{url}


% Macros for this document

\newcommand{\absfont}{\sffamily}
\newcommand{\abs}[1]{{\absfont #1}}



\begin{document}

\title{mit, the Mite translator}
\author{Reuben Thomas\\\url{rrt@sc3d.org}\\\url{http://sc3d.org/rrt/}}
\date{3rd April 2001}
\maketitle



\section{Introduction}

mit, the \textbf{Mi}te \textbf{t}ranslator, is written in ISO~C, and can be
used stand-alone or embedded in other programs.


\subsection{License}

mit is supplied as is, under BSD license:

\begin{quote}
Copyright \copyright 2001 Reuben Thomas

Permission is hereby granted, free of charge, to any person obtaining
a copy of this software and associated documentation files (the
``Software''), to deal in the Software without restriction, including
without limitation the rights to use, copy, modify, merge, publish,
distribute, sublicense, and/or sell copies of the Software, and to
permit persons to whom the Software is furnished to do so, subject to
the following conditions:

The above copyright notice and this permission notice shall be
included in all copies or substantial portions of the Software.

THE SOFTWARE IS PROVIDED ``AS IS'', WITHOUT WARRANTY OF ANY KIND,
EXPRESS OR IMPLIED, INCLUDING BUT NOT LIMITED TO THE WARRANTIES OF
MERCHANTABILITY, FITNESS FOR A PARTICULAR PURPOSE AND NONINFRINGEMENT.
IN NO EVENT SHALL THE AUTHORS OR COPYRIGHT HOLDERS BE LIABLE FOR ANY
CLAIM, DAMAGES OR OTHER LIABILITY, WHETHER IN AN ACTION OF CONTRACT,
TORT OR OTHERWISE, ARISING FROM, OUT OF OR IN CONNECTION WITH THE
SOFTWARE OR THE USE OR OTHER DEALINGS IN THE SOFTWARE.
\end{quote}


\subsection{System requirements}

mit should work on any big or little-endian system with an ISO~C
compiler and the standard libraries, provided:

\begin{itemize}
\item two's complement number representation is used
\item the compiler performs arithmetic right shifts on signed types
\item all pointers are the same size
\end{itemize}

\noindent I'd really like to know of any problems, whether because
your system doesn't meet these criteria, or because of a bug in mit.



\section{Interpreter}

The interpreter is contained in \verb|interp.c|. Some functions are
compiled twice, once with \verb|MITE_VERIFY| defined and once without.
The verified versions have some extra checks for validity of the
object code added, which are not needed during interpretation, once
the code has been checked; they have the prefix \verb|ver_| added to
their names by the \verb|MITE_PREFIX| macro.

The implementation-specific header is \verb|mite_i.h|. It contains the
implementation-dependent parameters and types.



\section{Utilities}

Mite uses a few utility macros. \verb|min()| and \verb|max()| are the
usual minimum and maximum operators; \verb|raise()| is a shorthand for
raising an exception. \verb|seta()| sets array element \verb|a[i]| to
\verb|e|$+$\verb|o|, incrementing \verb|i| and checking that it does
not exceed the size \verb|n| of the array. \verb|alloc()| is a wrapper
for \verb|malloc()| that raises an exception if it runs out of memory.



\section{Parameters}

\verb|mite_i.h| is included by \verb|mite.h| to define the
implementation-specific parameters and types. mit represents the Mite
word by \verb|intptr_t| and the unsigned word by \verb|uintptr_t|;
there are $8$ general registers by default (though this can be changed
by altering \verb|mite_g|'s definition), and the stack is descending.

The endianness of the current machine is held in \verb|endianness|,
which is simply a \verb|uint32_t| set to $1$. It is read via the macro
\verb|LITTLE_ENDIAN|, which looks at the first byte, which will be $1$
on a little-endian machine, and $0$ on a big-endian machine.



\section{Registers and stack}

Some macros are used to access registers and the stack, which depend
on \verb|R| being a register array. \verb|P_REG| is the register
number of the \abs{P} register; \verb|P| gives the value of \abs{P} as
a pointer to an instruction; \verb|Pp| gives the address of \verb|P|
as a pointer to \verb|miteObject|, which can be passed to
\verb|intg()| and \verb|num()| (section~\ref{numbers}) while reading
virtual code. \verb|setP()| is used to assign to \verb|P| (because
ANSI~C forbids casting l-values, and \verb|P| is actually a member of
an array of \verb|miteWord|s, not a \verb|uint32_t *|).

Similarly, \verb|S_REG| is the register number of \abs{S}, \verb|S|
gives \abs{S}'s value as a \verb|miteWord *|, and \verb|setS()| is
used to assign to \verb|S|. \verb|checkS()| is called by \verb|setS()|
and elsewhere to check that \abs{S}'s value is valid every time it is
updated; if not, \verb|mite_badS| is raised.

The stack is stored in a \verb|malloc|ed block, which is auto-extended
using the \verb|stack_extend()| macro, which is passed the
\verb|mite_State| whose stack it should extend (including a trailing
\verb|.| or \verb|->| to indicate the method of access to the
members). This macro is also used to initialise the stack, by setting
the state's \verb|S_base| to \verb|NULL|, which causes the call to
\verb|realloc()| to allocate a new block of memory. \verb|extendS| is
used during interpretation before each operation that pushes to the
stack, to extend it when full.



\section{Numbers} \label{numbers}

These two functions read numbers from a \verb|miteObject *|, updating
the pointer once the number has been read.

\verb|intg()| reads an encoded integer, gathering words until the top
bit is zero, and taking the sign from the first word. When verifying,
it takes as its second parameter a pointer to the end of the object
code and checks that it does not exceed this limit.

\verb|num()| reads a number with possible use of $e$, $s$, $w$ and rotation.
Because it calls \verb|intg()|, it too must have a verified version.



\section{Types}

\verb|miteProgram| holds all the data needed to interpret a program.
\verb|code| points to the start of the object code in the original
module, and \verb|cend| points to the end of the object code;
\verb|data| points to the copied out data, and \verb|dlen| is its
length; \verb|bl|, \verb|sl| and \verb|dl| point to arrays containing
the addresses of the labels, and \verb|b|, \verb|s| and \verb|d| give
the number of labels of each types.

\verb|mite_State_i| holds the base of the stack (\verb|S_base|), and
the amount of memory currently allocated for it (\verb|S_size|).



\section{Translation}

Before a program is interpreted, it is scanned to find all the labels,
which are stored in arrays for faster access during interpretation,
and the literal data is written into a separate memory block so that
it can easily be read and written. At the same time, the program is
verified to check that all the instructions and their operands are
valid.

\verb|mite_translate()| translates a program for interpretation,
verifying that it is valid in the process.

The number of labels of each type is now checked against the numbers
given in the header; the checks made while setting the values of
labels ensured that there can't be more labels than required, but
there might be fewer.

The current value of \verb|p| is stored as the value of the new label,
according to its type, then the number of labels of the current type
is checked to make sure it doesn't overflow.


\subsection{Verifying and copying data}

Next, the object code is scanned again. The data are copied out, and
the types of label operands are checked.

Finally, the program is returned.


\subsection{Reading data}

For a \abs{lit}, the number of operands, \verb|ops|, is read, and that
number of literal numbers is copied into the data block.

\abs{litl} can refer to the three types of label, whose addresses were
calculated in the previous pass over the object code.

\abs{space} requires the given number of words to be zero-filled.



\section{Execution}

\subsection{Operands}

Macros are provided to read operands of each type. The special type
\verb|_| is for unused operand places.

Given an instruction's opcode and operand types, \verb|OP()| declares
the case label and reads the instruction's operands. The \verb|break|
is placed before to terminate the previous case, as the code for the
instruction immediately follows. So that the first use of \verb|OP()|
in the switch does not cause a compiler warning, the \verb|default|
case is placed just before the first call of \verb|OP()|.


\subsection{Interpretation loop}

\verb|mite_run()| repeatedly loads, decodes and executes instructions
in an endless loop. The function is exited by a \abs{ret} instruction
when the stack is empty, or via \verb|raise()| if an error occurs.

Instructions are dispatched by a switch on the opcode.


\section{Instruction actions} \label{actions}

The default instruction actions are below. The \verb|OP| macro need
not do any checking of the arguments, as the code being interpreted is
known to be correct. Only run-time errors need be checked. Note that
there is no case for \abs{lab}, which therefore raises
\abs{badInstruction} if it is ever executed. The preprocessing step
sets the address of branch and subroutine labels to the instruction
after the \abs{lab} instruction, so no \abs{lab} instruction should
ever actually be executed.



\section{Standalone interpreter}

The \verb|main()| function allocates the stack, prepares the program,
and then starts it by calling \verb|mit_run()|, whose return value is
used as mit's exit code.



\section{Embedding mit}

mit can be embedded in a C program by compiling \verb|interp_main.c|
into the program. Since the interpreter has no global state, it can be
compiled as a shared library, and is thread-safe.



\section{Acknowledgements}

mit draws on implementation techniques developed for my earlier Beetle
virtual processor~\cite{beetledis}. Martin Richards introduced me to
virtual machine interpreter implementation techniques via his Cintcode
system~\cite{cintweb}.



\bibliographystyle{plain}
\bibliography{rrt,vm}

\end{document}
